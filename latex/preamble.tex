\documentclass[11pt, a4paper]{article}
% \usepackage[T1]{fontenc}
\usepackage{geometry}
\usepackage{amsmath, amssymb, bm}
\usepackage{accents}
\usepackage{blkarray}
\usepackage{enumerate}
\usepackage{physics, siunitx}
\usepackage{hyperref}
\usepackage{graphicx}

\geometry{margin=1.5cm}
\sisetup{separate-uncertainty=true, exponent-product=\cdot, range-units=single}
\hypersetup{colorlinks=true, linkcolor=blue, urlcolor=cyan}

\newcommand{\diff}{\mathop{}\!\mathrm{d}}
\newcommand{\TODO}[1]{{\textbf{TODO:} {\color{red} #1}}}

% \documentclass[12pt]{article}
% \usepackage[letterpaper]{geometry} 
%% \usepackage[T1]{fontenc} 
%% \usepackage{times} 
\usepackage{lmodern}
%% \usepackage{cmbright}
% \geometry{top=1.0in, bottom=1.0in, left=1.0in, right=1.0in}
\usepackage{setspace}
\usepackage[skip=10pt, indent=0pt]{parskip}
\onehalfspacing
%% \usepackage[colorinlistoftodos]{todonotes} \usepackage{amsmath}%
% \usepackage{MnSymbol}
% Tabular environment but better(coherent page breaking)
\usepackage{supertabular} 
\usepackage{hyperref}
% algorithm package for pseudocode blocks
% \usepackage{algorithm2e}

% --------------------------------------------- %
% Environment for escaping latex commands
% --------------------------------------------- %

\newenvironment{simplechar}{%
        \catcode`\^=12
        \catcode`\$=12
        \catcode`~=12
}{}

\newenvironment{tabulartiny}{
        \tiny
}{}


% --------------------------------------------- %
% Minted for Code Blocks
% --------------------------------------------- %

\usepackage{listings}
\usepackage[dvipsnames]{xcolor}
\usepackage{textcomp}
\usepackage[cache=false,newfloat]{minted}
\usepackage{tcolorbox, etoolbox}
\usepackage{caption}
\newcommand{\tqs}{\textquotesingle}
\newcommand{\ttilde}{{\raise.17ex\hbox{$\scriptstyle\sim$}}}
\tcbuselibrary{minted,skins}

% Caption setup, mainly remove numbering in captions
\captionsetup{labelformat=empty,singlelinecheck=false} 

% \textbf{} doesnt work in minted ... use \bf{}
\setminted[text]{%
    linenos,
    style=xcode,
    obeytabs=true,
    tabsize=2,
    numbersep=10pt,
    breaklines,
    escapeinside=||,
    mathescape=true,
    samepage,
    firstnumber=1,%
}

\setminted[java]{%
    linenos,
    style=tango,
    obeytabs=true,
    tabsize=2,
    numbersep=10pt,
    breaklines,
    escapeinside=||,
    mathescape=true,
    samepage,
    firstnumber=1,%
}

% Caption for code blocks
\newenvironment{capblock}{\captionsetup{type=listing}}{}
\SetupFloatingEnvironment{listing}{name=}

% Line numbers colors
\renewcommand{\theFancyVerbLine}{\textcolor[HTML]{000000}{\footnotesize\arabic{FancyVerbLine}}}

% Set start counter for line numbers in minted environment
%\renewcommand{\theFancyVerbLine}{%
%    \ifnum\value{FancyVerbLine}=0
%        \setcounter{FancyVerbLine}{0}
%    \else
%        \arabic{FancyVerbLine}%
%    \fi
%}

% --------------------------------------------- %
% Inkscape-figures script configuration
% --------------------------------------------- %

\usepackage{import}
\usepackage{xifthen}
\usepackage{pdfpages}
\usepackage{transparent}
\usepackage{float}

\newcommand{\incfig}[1]{%
    \def\svgwidth{\columnwidth}
    \import{./figures/}{#1.pdf_tex}
}

% --------------------------------------------- %
% Code/Algorithm Block Environments
% --------------------------------------------- %
\usepackage{mdframed}

% Commands for Pseudo-Code Environments
\newcommand{\ctitle}{\color{Aquamarine!100}{  }} 

\newcommand{\cfor}{\color{Aquamarine!100}{\bf{\textit{for}}}} 
\newcommand{\cin}{\color{Aquamarine!100}{\bf{\textit{in}}}} 
\newcommand{\cto}{\color{Aquamarine!100}{\bf{\textit{to}}}} 
\newcommand{\cwhile}{\color{Aquamarine!100}{\bf{\textit{while}}}} 

\newcommand{\cif}{\color{Aquamarine!100}{\bf{\textit{if}}}} 
\newcommand{\celse}{\color{Aquamarine!100}{\bf{\textit{else}}}} 
\newcommand{\cthen}{\color{Aquamarine!100}{\bf{\textit{then}}}} 

\newcommand{\clet}{\color{Aquamarine!100}{\bf{\textit{let}}}} 
\newcommand{\cand}{\color{Aquamarine!100}{\bf{\textit{and}}}} 

\newcommand{\creturn}{\color{Aquamarine!100}{\bf{\textit{return}}}} 

\newenvironment{pseudo}[2]
    {
    \VerbatimEnvironment
    \begin{mdframed}[
         frametitle={\color{black}{\large Pseudo-Code Algorithm}},
         frametitlebackgroundcolor = Dandelion!100,
         linecolor = Dandelion!100,
         topline = false,
         rightline = false,
         leftline = false,
         bottomline = true,
         linewidth = 5pt,%
    ]
    \begin{capblock}
    \vspace{10pt}
    \captionof{listing}{Input: #1}
    \vspace{-10pt}
    \captionof{listing}{Output: #2}
    \begin{minted}[fontfamily=lmodern]{R}%
    }
    {
     \end{minted}
     \end{capblock}
     \end{mdframed}
}
 
\newenvironment{code}[2]
    {
    \vspace{5pt}
    \VerbatimEnvironment
    \begin{mdframed}[
        frametitle={\color{black}{}},
        frametitlebackgroundcolor = White!50,
        linecolor = White!50,
        topline = false,
        rightline = false,
        leftline = false,
        bottomline = false,
        linewidth = 10pt,%
    ]
    \begin{capblock}
    \vspace{5pt}
    \captionof{listing}{#1}
    \begin{minted}{#2}%
    }
    {
    \end{minted}
    \end{capblock}
    \end{mdframed}
}

% --------------------------------------------- %
% ANALYZE PSEUDOCODE/CODE ENVIRONMENTS
% --------------------------------------------- %

% Pseudocode No Frame
\newenvironment{pseudonf}[2]
    {
    \VerbatimEnvironment
    \begin{capblock}
    \captionof{listing}{#1}
    \captionof{listing}{#2}
        \begin{minted}[fontfamily=lmodern, xleftmargin=40pt]{text}%
    }
    {
    \end{minted}
    \end{capblock}
}

% Code block no frame
\newenvironment{codenf}[2]
    {
    \VerbatimEnvironment
    \begin{capblock}
    \captionof{listing}{#1}
    \begin{minted}[xleftmargin=30pt]{#2}%
    }
    {
    \end{minted}
    \end{capblock}
}

% mdframe
\newenvironment{frm}[1]
    {
    \begin{mdframed}[
        frametitle={\color{black}{\large #1}},
        frametitlebackgroundcolor = Dandelion!100,
        linecolor = Dandelion!100,
        topline = false,
        rightline = false,
        leftline = false,
        bottomline = true,
        linewidth = 5pt,%
    ]
    }
    {
    \end{mdframed} 
}

% Black Box
\newenvironment{frmbox}
    {
    \bigskip
    \begin{mdframed}[
        linecolor = Black!100,
        topline = true,
        rightline = true,
        leftline = true,
        bottomline = true,
        linewidth = 1pt,%
    ]
    }
    {
    \end{mdframed} 
}

% --------------------------------------------- %
% Math Environments
% --------------------------------------------- %

% THEOREM
\newenvironment{theorem}[1]
    {
        \vspace*{10pt}
        \begin{mdframed}[
        frametitle={\color{Orange!100}{\large Theorem #1}},
        frametitlebackgroundcolor = white,
        linecolor = Orange!100,
        topline = false,
        rightline = false,
        leftline = true,
        bottomline = false,
        linewidth = 5pt,%
    ]
    }
    {
    \end{mdframed}
}

% LEMMA
\newenvironment{lemma}[1]
    {
    \vspace*{10pt}
    \begin{mdframed}[
        frametitle={\color{RedOrange!100}{\large Lemma #1}},
        frametitlebackgroundcolor = white,
        linecolor = RedOrange!100,
        topline = false,
        rightline = false,
        leftline = true,
        bottomline = false,
        linewidth = 5pt,%
    ]
    }
    {
    \end{mdframed}
}

% COROLLARY
\newenvironment{corollary}[1]
    {
    \vspace*{10pt}
    \begin{mdframed}[%
        frametitle={\color{SpringGreen!100}{\large Corollary #1}},
        frametitlebackgroundcolor = white,
        linecolor = SpringGreen!100,
        topline = false,
        rightline = false,
        leftline = true,
        bottomline = false,
        linewidth = 5pt,%
    ]
    }
    {
    \end{mdframed}
}

% DEFINITIONS
\newenvironment{definition}[1]
    {
    \vspace*{10pt}
    \begin{mdframed}[
        frametitle={\color{Blue!100}{\large Definition #1}},
        frametitlebackgroundcolor = white,
        linecolor = Blue!100,
        topline = false,
        rightline = false,
        leftline = true,
        bottomline = false,
        linewidth = 5pt,%
    ]
    }
    {
    \end{mdframed}
}

% AXIOM
\newenvironment{axiom}[1]
    {
    \vspace*{10pt}
    \begin{mdframed}[%
        frametitle={\color{RoyalPurple!100}{\large Axiom #1}},%
        frametitlebackgroundcolor = white,
        linecolor = RoyalPurple!100,
        topline = false,
        rightline = false,
        leftline = true,
        bottomline = false,
        linewidth = 5pt,%
    ]
    }
    {
    \end{mdframed}
}

% PROPERTY
\newenvironment{property}[1]
    {
    \vspace*{10pt}
    \begin{mdframed}[%
        frametitle={\color{Cyan!100}{\large Property #1}},%
        frametitlebackgroundcolor = white,
        linecolor = Cyan!100,
        topline = false,
        rightline = false,
        leftline = true,
        bottomline = false,
        linewidth = 5pt,%
    ]
    }
    {
    \end{mdframed}
}

% PROPOSITION
\newenvironment{proposition}[1]
    {
    \vspace*{10pt}
    \begin{mdframed}[%
        frametitle={\color{BrickRed!100}{\large Proposition #1}},%
        frametitlebackgroundcolor = white,
        linecolor = BrickRed!100,
        topline = false,
        rightline = false,
        leftline = true,
        bottomline = false,
        linewidth = 5pt,%
    ]
    }
    {
    \end{mdframed}
}

% EXAMPLE
\newenvironment{example}[1]
    {
    \vspace*{10pt}
    \begin{mdframed}[%
        frametitle={\color{ForestGreen!100}{\large Example #1}},
        frametitlebackgroundcolor = white,
        linecolor = ForestGreen!100,
        topline = false,
        rightline = false,
        leftline = true,
        bottomline = false,
        linewidth = 5pt,%
    ]
    }
    {
    \end{mdframed}
}

% PROOF
\newenvironment{proof}[1]
    {
    \vspace*{10pt}
    \begin{mdframed}[
        frametitle={\color{YellowOrange!100}{\large Proof #1}},
        frametitlebackgroundcolor = white,
        linecolor = YellowOrange!100,
        topline = false,
        rightline = false,
        leftline = true,
        bottomline = false,
        linewidth = 5pt,%
    ]
    }
    {
    \end{mdframed}
}

% NOTE
\newenvironment{note}
    {
    \vspace*{10pt}
    \begin{mdframed}[
        frametitle={Note:},
        frametitlebackgroundcolor = white,
        linecolor = Gray!50,
        topline = false,
        rightline = false,
        leftline = true,
        bottomline = false,
        linewidth = 5pt,%
    ]
    }
    {    
    \end{mdframed}
}

% QUESTION
\newenvironment{question}
    {
    \vspace*{10pt}
    \begin{mdframed}[
        frametitle={Question:},
        frametitlebackgroundcolor = white,
        linecolor = Purple!50,
        topline = false,
        rightline = false,
        leftline = true,
        bottomline = false,
        linewidth = 5pt,%
    ]
    }
    {    
    \end{mdframed}
}

% --------------------------------------------- %
% Misc Config
% --------------------------------------------- %

\newcommand{\Z}{\mathbf{Z}}
\newcommand{\R}{\mathbf{R}}
\newcommand{\Q}{\mathbf{Q}}
\newcommand{\N}{\mathbf{N}}







% \setcounter{section}{1}
